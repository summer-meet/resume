%based on the guide at http://www.cv-templates.info/2009/03/professional-cv-latex/
\documentclass[a4paper,10pt]{article}

%A Few Useful Packages
\usepackage{marvosym}
\usepackage{fontspec} 					%for loading fonts
\usepackage{xunicode,xltxtra,url,parskip} 	%other packages for formatting
\usepackage{setspace} %spacing
\RequirePackage{color,graphicx}
\usepackage[usenames,dvipsnames]{xcolor}
\usepackage[big]{layaureo} 				%better formatting of the A4 page
% an alternative to Layaureo can be ** \usepackage{fullpage} **
\usepackage{supertabular} 				%for Grades
\usepackage{titlesec}					%custom \section
\usepackage{paralist}
\usepackage{geometry}                     %for setting the margin
\usepackage{multirow}
\usepackage{array}
\usepackage{xcolor}
%Setup hyperref package, and colours for links
\usepackage{hyperref}
\usepackage{marvosym} %symbol
\definecolor{linkcolour}{rgb}{0,0.2,0.6}
\hypersetup{colorlinks,breaklinks,urlcolor=linkcolour, linkcolor=linkcolour}

%FONTS
\defaultfontfeatures{Mapping=tex-text}
\setmainfont[ Path = fonts/, BoldFont = Fontin-Bold, BoldItalicFont = Fontin-Bold, ItalicFont = Fontin-Italic, SmallCapsFont = Fontin-SmallCaps]{Fontin-Regular}%SmallCaps]{Fontin-Regular}
%\setmainfont[Path = fonts/, BoldFont = LinLibertine_RB.otf, ItalicFont = LinLibertine_RI.otf, BoldItalicFont = LinLibertine_RBI.otf, SlantedFont = LinLibertine_aRL.otf, BoldSlantedFont = LinLibertine_aBL.otf, SmallCapsFont = LinLibertine_aS.otf]{LinLibertine_R.otf}
%\setmainfont[ Path = fonts/, BoldFont = GILB, BoldItalicFont = GILBI, ItalicFont = GILI, SmallCapsFont = GIL]{GIL}

%CV Sections inspired by:
%http://stefano.italians.nl/archives/26
%\titleformat{\section}{\Large\scshape\raggedright}{}{0em}{}[\titlerule]%\scshape\raggedright}{}{0em}{}[\titlerule]
%\titlespacing{\section}{0pt}{3pt}{3pt}
%Tweak a bit the top margin
%\addtolength{\voffset}{-1.3cm}

%Italian hyphenation for the word: ''corporations''
%\hyphenation{im-pre-se}


%............

%\tolerance=1000
%\emergencystretch=\maxdimen
%\hyphenpenalty=10%10000
%\hbadness=10%10000
\hyphenchar\font=-1

%........



\newgeometry{left=1in,right=1in,top=1in, bottom=1in}  %20140829 top 1.5-->2.5
%--------------------BEGIN DOCUMENT----------------------
\begin{document}
\thispagestyle {empty}
\pagestyle{empty}
\begin{spacing}{1.5}
%\pagestyle{empty} % non-numbered pages

%:\tiny-6pt、\scritpsize-8pt、\footnotesize-10pt、\small-11pt、

%\normalsize-12pt、\large-14pt、\Large-17pt、\LARGE-20、\huge-25pt、\Huge-25pt。

\font\fb="[cmr10]" %for use with \LaTeX command

\begin{center} {\LARGE {\textsc{Statement of Purpose}}     }  \end{center}%\textbf{{\Huge{Q}\Large{ICHEN} \Huge{S}\Large{ONG}}%}
\vspace{1em}
\begin{center} {\large {Qichen Song}     }  \end{center}
%-------------------Demarcation---------

\vspace{1.5em}
My undergraduate study and research experience opened the door to the amazing nano-world for me and ignited my interest in scientific research. I enjoy the process of cracking problems by innovative thinking and I acquire great pleasure from proposing my own solutions. \\
\\
I am desirous to conduct research on nanoscale thermal transport, especially on phonon transport and phonon manipulation during my PhD studies. Innovative nanostructured materials have attracted extensive scientific interest due to their unique thermal properties. For traditional bulk materials, we cannot significantly alter their thermal properties. But at nanoscale, we can manage and modulate the thermal properties of materials as we want. Nanomaterials actually give us a far better chance to produce next generation of thermoelectric devices to resolve energy shortage problems. But current energy conversion efficiency of the devices is not satisfying for commercial uses due to lack of a detailed understanding of thermal transport in nanostructures. I believe my research will help to bring that bright future closer to us.\\
\\
My undergraduate study helped lay a solid foundation for my future research. My major courses such as thermodynamics (91/100), heat transfer (95/100) and fluid dynamics (99/100) helped me to understand basic rules and principles of energy conversion and thermal management. Meanwhile, I acquired solid skills in coding for high precision scientific computation from the course of numerical methods (100/100) and C++ programming design (95/100). I came to Nano Heat Group, led by Professor Nuo Yang, to acquire further knowledge of mechanism of nanoscale thermal transport.\\
\\
My research experience helped me in improving my ability of designing proper model and analyzing problems of nanoscale thermal transport. I focused on finding ways to reduce thermal conductivity of graphene for its potential thermoelectric application. Folding has been proved to be a useful way to manipulate the thermal conductivity of graphene nanoribbon. However, given the fact that there exists a strong dependence between graphene’s thermal conductivity and its size, the large-area folded graphene may be a different case and thus should be investigated. I successfully applied my own FORTRAN program to set atom’s initial position, initial velocity distribution and atom reciprocal potentials explicitly and reasonably. I calculated cases of a series of different sizes to obtain thermal conductivity independent with size. Challenging was that computers cannot deal with some cases of large simulation scale. I subtly optimized the code and accelerated the computing process significantly, which successfully solved this issue. And my results clearly show that thermal conductivity decreases significantly with increasing number of folds and stronger substrate effect, presumably caused by the enhanced phonon scattering in the folded area. This proves that even for large-area graphene with folds, folding still greatly contributes to reducing the thermal conductivity. This work is being enriched and will be published soon. From this work, I have learned every detail of MD simulation and how to implement ideas in my mind via programming.\\
\\
Capability of collaboration helped me even more prepared for my future research. In the research on coupling between different phonon modes in graphene, Dr. Meng An came up with theory model at first and I built the simulation model to verify its validity. Then we discussed how to modify the model based on simulation results. We did this feedback process time after time for months to make sure we finally obtained an appropriate model which can rationally quantify coupling. We kept doing double-check of each other’s work to remove errors. It was teamwork that promoted the research greatly.\\
\\
I seek advanced knowledge and skills I need from the graduate program to systematically conduct my research. I am looking forward to opportunities to do experiments to study thermal properties directly. Among the talented faculties of MAE department at UCLA, Dr. Yongjie Hu has done an impressive job on investigating thermal transport by ultrafast spectroscopy. I will be honored and delighted to work with him. After my graduation I will definitely keep pursuing my academic career in the research on nanoscale heat transfer for I believe that Rome was not built in one day: to convert my expertise into valuable scientific and social contribution, years of continuous effort is a must.\\

\end{spacing}
\end{document}
